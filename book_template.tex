% Options for packages loaded elsewhere
\PassOptionsToPackage{unicode}{hyperref}
\PassOptionsToPackage{hyphens}{url}
%
\documentclass[
]{book}
\usepackage{amsmath,amssymb}
\usepackage{lmodern}
\usepackage{ifxetex,ifluatex}
\ifnum 0\ifxetex 1\fi\ifluatex 1\fi=0 % if pdftex
  \usepackage[T1]{fontenc}
  \usepackage[utf8]{inputenc}
  \usepackage{textcomp} % provide euro and other symbols
\else % if luatex or xetex
  \usepackage{unicode-math}
  \defaultfontfeatures{Scale=MatchLowercase}
  \defaultfontfeatures[\rmfamily]{Ligatures=TeX,Scale=1}
\fi
% Use upquote if available, for straight quotes in verbatim environments
\IfFileExists{upquote.sty}{\usepackage{upquote}}{}
\IfFileExists{microtype.sty}{% use microtype if available
  \usepackage[]{microtype}
  \UseMicrotypeSet[protrusion]{basicmath} % disable protrusion for tt fonts
}{}
\makeatletter
\@ifundefined{KOMAClassName}{% if non-KOMA class
  \IfFileExists{parskip.sty}{%
    \usepackage{parskip}
  }{% else
    \setlength{\parindent}{0pt}
    \setlength{\parskip}{6pt plus 2pt minus 1pt}}
}{% if KOMA class
  \KOMAoptions{parskip=half}}
\makeatother
\usepackage{xcolor}
\IfFileExists{xurl.sty}{\usepackage{xurl}}{} % add URL line breaks if available
\IfFileExists{bookmark.sty}{\usepackage{bookmark}}{\usepackage{hyperref}}
\hypersetup{
  pdftitle={Bioconductor Guidelines for Mentors and Mentees},
  pdfauthor={Kevin Rue-Albrecht},
  hidelinks,
  pdfcreator={LaTeX via pandoc}}
\urlstyle{same} % disable monospaced font for URLs
\usepackage{longtable,booktabs,array}
\usepackage{calc} % for calculating minipage widths
% Correct order of tables after \paragraph or \subparagraph
\usepackage{etoolbox}
\makeatletter
\patchcmd\longtable{\par}{\if@noskipsec\mbox{}\fi\par}{}{}
\makeatother
% Allow footnotes in longtable head/foot
\IfFileExists{footnotehyper.sty}{\usepackage{footnotehyper}}{\usepackage{footnote}}
\makesavenoteenv{longtable}
\usepackage{graphicx}
\makeatletter
\def\maxwidth{\ifdim\Gin@nat@width>\linewidth\linewidth\else\Gin@nat@width\fi}
\def\maxheight{\ifdim\Gin@nat@height>\textheight\textheight\else\Gin@nat@height\fi}
\makeatother
% Scale images if necessary, so that they will not overflow the page
% margins by default, and it is still possible to overwrite the defaults
% using explicit options in \includegraphics[width, height, ...]{}
\setkeys{Gin}{width=\maxwidth,height=\maxheight,keepaspectratio}
% Set default figure placement to htbp
\makeatletter
\def\fps@figure{htbp}
\makeatother
\setlength{\emergencystretch}{3em} % prevent overfull lines
\providecommand{\tightlist}{%
  \setlength{\itemsep}{0pt}\setlength{\parskip}{0pt}}
\setcounter{secnumdepth}{5}
\usepackage{booktabs}
\ifluatex
  \usepackage{selnolig}  % disable illegal ligatures
\fi
\usepackage[]{natbib}
\bibliographystyle{apalike}
\newlength{\cslhangindent}
\setlength{\cslhangindent}{1.5em}
\newlength{\csllabelwidth}
\setlength{\csllabelwidth}{3em}
\newenvironment{CSLReferences}[2] % #1 hanging-ident, #2 entry spacing
 {% don't indent paragraphs
  \setlength{\parindent}{0pt}
  % turn on hanging indent if param 1 is 1
  \ifodd #1 \everypar{\setlength{\hangindent}{\cslhangindent}}\ignorespaces\fi
  % set entry spacing
  \ifnum #2 > 0
  \setlength{\parskip}{#2\baselineskip}
  \fi
 }%
 {}
\usepackage{calc}
\newcommand{\CSLBlock}[1]{#1\hfill\break}
\newcommand{\CSLLeftMargin}[1]{\parbox[t]{\csllabelwidth}{#1}}
\newcommand{\CSLRightInline}[1]{\parbox[t]{\linewidth - \csllabelwidth}{#1}\break}
\newcommand{\CSLIndent}[1]{\hspace{\cslhangindent}#1}

\title{Bioconductor Guidelines for Mentors and Mentees}
\author{Kevin Rue-Albrecht}
\date{2022-02-28}

\begin{document}
\maketitle

{
\setcounter{tocdepth}{1}
\tableofcontents
}
\hypertarget{about-bioconductor}{%
\chapter*{About Bioconductor}\label{about-bioconductor}}
\addcontentsline{toc}{chapter}{About Bioconductor}

\begin{center}\includegraphics[height=200px]{https://raw.githubusercontent.com/Bioconductor/BiocStickers/master/Bioconductor/Bioconductor} \end{center}

\href{https://bioconductor.org/}{\emph{Bioconductor}} provides tools for the analysis and comprehension of high-throughput genomic data \citep{orchestrating2015}.

\href{https://bioconductor.org/}{\emph{Bioconductor}} uses the statistical programming language \citep{r2021}, and is open source and open development. It has two releases each year, and an active user community.

\href{https://bioconductor.org/}{\emph{Bioconductor}} is also available as \href{https://bioconductor.org/help/docker/}{Docker} images.

\hypertarget{about-the-mentorship-program}{%
\chapter{About the mentorship program}\label{about-the-mentorship-program}}

The \href{https://bioconductor.org/}{\emph{Bioconductor}} website includes a page including a description of the \href{https://bioconductor.org/developers/new-developer-program/}{New Developer Program},
including links to apply as a mentor or a mentee.

\hypertarget{part-resources}{%
\part{Resources}\label{part-resources}}

\hypertarget{resources-overview}{%
\chapter*{Overview}\label{resources-overview}}
\addcontentsline{toc}{chapter}{Overview}

The following sections contain teachable materials for the mentorship programme.
For more technical information about Bioconductor package requirements inspected during the formal review process for new Bioconductor, please consult the \href{http://contributions.bioconductor.org/}{Bioconductor Package Guidelines for Developers and Reviewers}.

Mentors are encouraged to update and contribute new materials in this book from their experience.

Mentees are encouraged to provide feedback indicating areas of interest and topics that should be expanded or clarified.

Contributions to this book are expected through \href{https://github.com/kevinrue/bioc-mentorship-docs/pulls}{pull requests}.
Suggestions and feedback are expected through \href{https://github.com/kevinrue/bioc-mentorship-docs/issues}{issues}.

\hypertarget{community}{%
\chapter{Welcome to the Bioconductor Community}\label{community}}

The Bioconductor community comprises users and developers, and the goal of the mentorship programme is to make the transition from user to developer as smooth and enjoyable as possible.

A number of platforms are available for community members to interact with each other:

\begin{itemize}
\tightlist
\item
  \href{https://bioc-community.herokuapp.com/}{Slack}
\item
  \href{https://twitter.com/Bioconductor}{Twitter}
\item
  \href{https://support.bioconductor.org/}{Bioconductor Support Site}
\end{itemize}

Teaching materials and resources are also available to learn about the Bioconductor project:

\begin{itemize}
\tightlist
\item
  \href{https://carpentries-incubator.github.io/bioc-project/}{The Bioconductor project (Carpentries lesson, in development)}
\item
  \href{https://www.huber.embl.de/msmb/course_spring_2020/index.html}{Modern Statistics for Modern Biology}
\end{itemize}

\hypertarget{style-guide}{%
\chapter{The Bioconductor Coding Style Guide}\label{style-guide}}

The Bioconductor project maintains a coding style guide that promotes good practices for package readability, and guides the review process for new package contributions.

In particular, encouraging a community of developers to respect a shared coding style guide facilitate contributions and bug fixes to existing packages from community members who may not have been involved in the initial package development, but recognize familiar coding patterns.

The current version of the official Bioconductor coding style is available from the \href{https://bioconductor.org/developers/how-to/coding-style/}{Bioconductor website}.
Additional information is also available from the \href{http://contributions.bioconductor.org/}{Bioconductor Package Guidelines for Developers and Reviewers}.

\hypertarget{s4}{%
\chapter{Bioconductor and S4 classes}\label{s4}}

Bioconductor widely uses the S4 object system.

In particular, developers are strongly encouraged to re-use existing \href{https://bioconductor.org/developers/how-to/commonMethodsAndClasses/}{Common Bioconductor Methods and Classes} where possible.

The need for new classes may occasionally be identified (e.g., new technology or data type).
Often, new classes are derived from existing classes (i.e., inheriting slots from a parent class), while adding new functionality under the form of new slots and/or methods.

\begin{quote}
Contribute:

Link to conference workshops.
\end{quote}

\hypertarget{git-github}{%
\chapter{Git and GitHub}\label{git-github}}

The Bioconductor project uses the \href{https://git-scm.com/}{Git} software for version control of its central code repository.

\begin{center}\includegraphics[height=500px]{https://pbs.twimg.com/media/E1RiGCWWQAAPVy7?format=png&name=medium} \end{center}

Many developers also use the \href{https://github.com/}{GitHub} website to develop, test, and manage their packages source code outside the Bioconductor build system.
Once a package was accepted on the Bioconductor repository, developers may continue to use GitHub, but should remember to regularly push their changes to the Bioconductor repository in order to publish new versions of their packages through the Bioconductor installation machinery.

It is important to note that packages released and installed through the Bioconductor machinery (e.g., \texttt{BiocManager::install(\textquotesingle{}Biobase\textquotesingle{})}) come from the central Bioconductor repository, unless the name of the package to install is provided in the format of a gitHub repository \texttt{user/repo} (e.g., \texttt{BiocManager::install(\textquotesingle{}Bioconductor/Biobase\textquotesingle{})}).

The \href{https://code.bioconductor.org/}{Bioc::CodeExplore} app was developed by \href{https://github.com/grimbough}{Mike Smith} as a convenient web-interface to efficiently browse and search the Bioconductor Git repository.

\hypertarget{r-package}{%
\chapter{Making an R package}\label{r-package}}

Bioconductor packages are first and foremost R packages.
Before considering guidelines specific to the Bioconductor project, developers should familiarize themselves with best practices for R packages generally.

The following resources form a trove of information for new and experienced package developers:

\begin{itemize}
\tightlist
\item
  The \href{https://r-pkgs.org/}{R Packages} book
\item
  The \href{https://devguide.ropensci.org/building.html}{rOpenSci Packages: Development, Maintenance, and Peer Review} book
\end{itemize}

\hypertarget{bioc-package}{%
\chapter{Making an Bioconductor package}\label{bioc-package}}

\begin{quote}
Contribute!

Introduce and discuss the links below.
\end{quote}

\begin{itemize}
\tightlist
\item
  \href{TODO}{Link to Kayla's from Bioc2020}
\item
  \href{TODO}{BJ's Boston Meetup}
\item
  \href{TODO}{Saskia's video}
\item
  \href{https://github.com/kevinrue/BiocPackageSofwareTemplate}{Template Bioconductor package (in development)}
\item
  \href{https://github.com/kevinrue/bioc_package_guide}{Bioconductor Package Guidelines for Developers and Reviewers}
\item
  \href{https://github.com/SaskiaFreytag/making_bioconductor_pkg}{Building a Bioconductor package using RStudio}
\item
  \href{https://github.com/lshep/MakeAPackage}{lshep/MakeAPackage}
\end{itemize}

\hypertarget{biocthis}{%
\section{biocthis}\label{biocthis}}

The \emph{\href{https://bioconductor.org/packages/3.13/biocthis}{biocthis}} package provide convenient utilities to automate package and project setup for Bioconductor packages.

\begin{itemize}
\tightlist
\item
  \href{https://lcolladotor.github.io/biocthis/reference/use_bioc_pkg_templates.html}{\texttt{biocthis::use\_bioc\_pkg\_templates()}} creates four \texttt{dev/*.R} scripts that guide you in the process of setting up an RStudio project for a Bioconductor-friendly R package.
\item
  2021-01-28 \href{https://youtu.be/3fLNsLchPnI}{two minute video} in Spanish for ConectaR 2021. \href{https://speakerdeck.com/lcolladotor/biocthis-conectar2021}{Slides in Spanish}.
\item
  2020-11-05 \href{https://speakerdeck.com/lcolladotor/biocthis-tab}{slides in English} for the Bioconductor Technical Advisory Board.
\item
  2020-09-10 \href{https://youtu.be/aMTxkYsM-8o}{55 minute video in English}. \href{https://speakerdeck.com/lcolladotor/making-bioc-packages-with-biocthis}{Slides in English}.
\end{itemize}

\hypertarget{bioc-review}{%
\chapter{The Bioconductor Package Review Process}\label{bioc-review}}

The Bioconductor package review process provides feedback encouraging best practices in terms of:

\begin{itemize}
\tightlist
\item
  package size
\item
  re-use of existing Bioconductor classes and methods
\item
  general respect of the Bioconductor package guidelines and coding style
\end{itemize}

\begin{quote}
Contribute!

Discuss what to expect as far as the submission process and how feedback works.
\end{quote}

In particular, developers are encouraged to consult the \href{http://contributions.bioconductor.org/}{Bioconductor Package Guidelines for Developers and Reviewers}.

\hypertarget{bioc-workshop}{%
\chapter{Making a Bioconductor workshop}\label{bioc-workshop}}

\begin{quote}
Contribute!

Introduce and discuss the links below.
\end{quote}

\begin{itemize}
\tightlist
\item
  The \href{http://app.orchestra.cancerdatasci.org/}{Orchestra} website.

  \begin{itemize}
  \tightlist
  \item
    Using \href{https://github.com/features/actions}{GitHub Actions}
  \item
    The \href{https://github.com/seandavi/BuildABiocWorkshop}{template workshop package}
  \end{itemize}
\item
  The \href{https://github.com/jdrnevich/BuildACarpentriesWorkshop}{BuildACarpentriesWorkshop} workshop
\end{itemize}

\hypertarget{maintenance}{%
\chapter{Maintaining a Bioconductor package}\label{maintenance}}

\begin{quote}
Contribute!

Introduce and discuss the points below.
\end{quote}

\begin{itemize}
\tightlist
\item
  Bioc credential
\item
  Single developer on the Bioconductor Git repository \texttt{git.bioconductor.org}
\item
  Multiple developers on \url{http://github.com/}
\end{itemize}

\hypertarget{publish}{%
\chapter{Publishing a Bioconductor package}\label{publish}}

\begin{quote}
Contribute!

Introduce and discuss the points below.
\end{quote}

\begin{itemize}
\tightlist
\item
  F1000 and other places to publish
\end{itemize}

\hypertarget{package-community}{%
\chapter{Build your package community}\label{package-community}}

\begin{quote}
Contribute!

Introduce and discuss the points and links below.
\end{quote}

\begin{itemize}
\tightlist
\item
  Supporting your users

  \begin{itemize}
  \tightlist
  \item
    Inclusive web design
  \item
    \href{https://uxdesign.cc/designing-for-all-a-detailed-guide-to-designing-for-disabilities-aa92f5dcc49b}{blog with suggestions}
  \item
    The \href{http://www.bioconductor.org/about/code-of-conduct/}{Bioconductor code of conduct}
  \end{itemize}
\item
  Connecting with other package developers

  \begin{itemize}
  \tightlist
  \item
    R Devel mailing list
  \item
    Slack
  \end{itemize}
\item
  Contributing to Bioconductor

  \begin{itemize}
  \tightlist
  \item
    GitHub
  \item
    Advisory boards (e.g., CAB, TAB)
  \end{itemize}
\item
  Supporting a global Bioconductor community

  \begin{itemize}
  \tightlist
  \item
    \href{https://carpentries.github.io/glosario/}{Carpentries glossario} provides translation of technical terms related to R and Python.
  \end{itemize}
\end{itemize}

\hypertarget{part-mentors}{%
\part{Mentors}\label{part-mentors}}

\hypertarget{mentors-overview}{%
\chapter*{Overview}\label{mentors-overview}}
\addcontentsline{toc}{chapter}{Overview}

The following sections contain information and advice to mentors.

\hypertarget{part-mentees}{%
\part{Mentees}\label{part-mentees}}

\hypertarget{mentee-overview}{%
\chapter*{Overview}\label{mentee-overview}}
\addcontentsline{toc}{chapter}{Overview}

The following sections contain information and advice to mentees.

\hypertarget{appendix-appendix}{%
\appendix}


\hypertarget{booknews}{%
\chapter{NEWS}\label{booknews}}

\hypertarget{version-1.0.0-2021-07-27}{%
\section{Version 1.0.0 (2021-07-27)}\label{version-1.0.0-2021-07-27}}

\begin{itemize}
\tightlist
\item
  Initial definition of the mentorship guidelines.
\end{itemize}

\hypertarget{refs}{}
\begin{CSLReferences}{0}{0}
\end{CSLReferences}

  \bibliography{book.bib,packages.bib}

\end{document}
